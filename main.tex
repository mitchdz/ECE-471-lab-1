\documentclass[12pt]{article}
\usepackage[margin=1in]{geometry}                % See geometry.pdf to learn the layout options. There are lots.
\geometry{letterpaper}                   % ... or a4paper or a5paper or ... 
%\geometry{landscape}                % Activate for for rotated page geometry
\usepackage[parfill]{parskip}    % Activate to begin paragraphs with an empty line rather than an indent

%%%%%%%%%%%%%%%%%%%%
\newcommand{\hide}[1]{}

\usepackage{natbib}
\usepackage{xcolor}
\usepackage{url}
\usepackage{hyperref}
\usepackage{mathtools}
\usepackage[utf8]{inputenc}


\hide{
\usepackage{amscd}
\usepackage{amsfonts}
\usepackage{amsmath}
\usepackage{amssymb}
\usepackage{amsthm}
\usepackage{cases}		 
\usepackage{cutwin}
\usepackage{enumerate}
\usepackage{enumitem}
\usepackage{epstopdf}
\usepackage{graphicx}
\usepackage{ifthen}
\usepackage{lipsum}
\usepackage{mathrsfs}	
\usepackage{multimedia}
\usepackage{wrapfig}
}
\bibliographystyle{humanbio}


\usepackage[utf8]{inputenc}

\newcommand{\itemlist}[1]{\begin{itemize}#1\end{itemize}}
\newcommand{\enumlist}[1]{\begin{enumerate}#1\end{enumerate}}
\newcommand{\desclist}[1]{\begin{description}#1\end{description}}
\newcommand\tab[1][0.5cm]{\hspace*{#1}}

\newcommand{\Answer}[1]{\begin{quote}{\color{blue}#1}\end{quote}}
\newcommand{\AND}{\wedge}
\newcommand{\OR}{\vee}
\newcommand{\ra}{\rightarrow}
\newcommand{\lra}{\leftrightarrow}

\title {{\bf ECE 471 Lab 1} \\
\large{Secret-Key Encryption Lab}}

\author{Mitchell Dzurick}
\date{2/3/2020}
\begin{document}

\maketitle

\tableofcontents

\clearpage


Secret Key Encryption Lab

Copyright © 2018 Wenliang Du, Syracuse University. The development of this document was partially funded by the National Science Foundation under Award No. 1303306 and 1718086. This work is licensed under a Creative Commons Attribution-NonCommercial- ShareAlike 4.0 International License. A human-readable summary of (and not a substitute for) the license is the following: You are free to copy and redistribute the material in any medium or format. You must give appropriate credit. If you remix, transform, or build upon the material, you must distribute your contributions under the same license as the original. You may not use the material for commercial purposes.

\section{Overview}

The learning objective of this lab is for students to get familiar with the concepts in the secret-key encryption. After finishing the lab, students should be able to gain a first-hand experience on encryption algorithms, encryption modes, paddings, and initial vector (IV). Moreover, students will be able to use tools and write programs to encrypt/decrypt messages. This lab covers the following topics:

    \begin{itemize}
        \item Secret-key encryption
        \item Substitution cipher and frequency analysis
        \item Encryption modes and paddings
        \item Programming using the crypto library
    \end{itemize}

\textbf{Lab Environment}. This lab has been tested on our pre-built Ubuntu 12.04 VM and Ubuntu 16.04 VM, both of which can be downloaded from the SEED website.

\clearpage

\section{Lab Tasks}
\subsection{Task 1: Frequency Analysis Against Monoalphabetic Substitution Cipher}
    
It is well-known that monoalphabetic substitution cipher (also known as monoalphabetic cipher) is not secure, because it can be subjected to frequency analysis. In this lab, you are given a cipher-text that is encrypted using a monoalphabetic cipher; namely, each letter in the original text is replaced by another letter, where the replacement does not vary (i.e., a letter is always replaced by the same letter during the encryption). Your job is to find out the original text using frequency analysis. It is known that the original text is an English article.

\subsubsection{Task 1: solution}




\end{document}
